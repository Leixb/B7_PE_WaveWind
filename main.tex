\input{preamble}

\geometry{margin=1in}

\title{
   Probabilidad y estadística - B7 \\
   \large 
   - Estudio del oleaje y el viento en Nazaré y Jaws -
}
\author{
  Aleix Boné \and
  Alex Herrero \and
  Albert Mercadé
}
\date{
  \today
}

\begin{document}
\maketitle
%\tableofcontents

\begin{abstract}
% resumen
% <= 250 palabras

En este trabajo se estudian dos factores muy importantes en el surf, la altura de las olas y la velocidad 
del viento en dos de los sitios más conocidos dentro de este deporte, Nazaré, Portugal y Jaws en Peahi, Hawái. 

Cogiendo una pequeña muestra de todos los datos, planteamos la hipótesis que la media de altura de olas
en Nazaré es mayor a las de Jaws, la cual confirmamos al hacer una prueba de hipótesis unilateral.
También calculamos cómo afecta la velocidad del viento a la altura de las olas en ambas localizaciones 
por separado, con un modelo lineal, determinando que existe una correlación positiva aunque no muy fuerte.

Finalmente, averiguamos que el efecto del viento no es igual en los dos \textit{surf spots}; en Nazaré
el efecto del viento es perceptible a cualquier velocidad mientras que en Jaws se necesita un mínimo de 
10 nudos aproximadamente para que el efecto sea apreciable.

\end{abstract}

\section{Introducción}%
\label{sec:introduccion}
% Justificación + objetivos + hipotesis
El gran interés de uno de los integrantes del grupo, Alex Herrero, en el mundo del surf nos ha llevado a investigar este apasionante tema y nos ha introducido al resto del grupo al mundo del surf y la oceanografía. Dentro del ámbito del surf hay muchos factores a tener en cuenta en cuanto a lo que se refiere a la ``ola perfecta''. Entre ellos, dos de los más importantes son el tamaño de la ola y la velocidad del viento. Es un tema complejo dado que los datos que vamos a estudiar se ven afectados por muchos otros factores; la marea, el \textit{fetch}\footnote{La longitud rectilínea máxima de una gran masa de agua superficial que es uniformemente afectada por la dirección y fuerza del viento} o la complexión del fondo marino litoral.

En este caso queremos comparar dos de los \textit{surf spots}\footnote{Lugar con olas surfeables} más famosos y con las olas más grandes del mundo a lo largo de la historia; Nazaré en Portugal y Jaws en Peahi, Hawái. Paralelamente observaremos qué relación tiene la velocidad de viento con el tamaño de las olas en ambos sitios.

Nuestra hipótesis es que la media de la altura de las olas en Nazaré será superior a la de Jaws, dado que en los últimos años el récord de ola más grande surfeada por el hombre se ha roto en Nazaré varias veces. Además, el fondo marino de Nazaré tiene un relieve mejor para la creación de olas gigantes. En cuanto a la relación entre la velocidad del viento y la altura de las olas, esperamos que la correlación sea positiva
puesto que la velocidad del viento es uno de los factores más importantes en la creación de las olas.

\section{Recogida de datos}%
\label{sec:recogida_de_datos}
% Recogida de datos + seleccion de datos
Para realizar nuestro análisis recogimos datos de viento y altura de olas de
dos localizaciones distintas. Escogimos dos playas conocidas por sus buenas
olas para hacer surf: Nazaré y Jaws.

Pudimos obtener datos de oleaje y viento del archivo histórico de \emph{WindGuru}\footnote{\url{https://www.windguru.cz/archive.php}}, un servicio especializado en previsiones del tiempo para windsurfistas y kitesurfistas basado en el modelo numérico GFS\footnote{\url{https://es.wikipedia.org/wiki/Global_Forecast_System}}. Con un \textit{script} de
Python\footnote{El código se puede ver en el anexo
  \ref{sec:codigo_extraccion_de_datos}} extrajimos datos de la altura de las
olas y viento en periodos de 3 horas de 2006 hasta hoy. En el caso de Nazaré
obtuvimos datos de 4377 días (35016 periodos de 3 horas) y en Jaws 3338 días
(26704 periodos de 3 horas).

Con el fin de simplificar los cálculos al hacer la prueba de hipótesis y la regresión lineal simple nos interesa reducir la cantidad de datos que utilizamos en nuestro estudio. Para ello, hemos decidido considerar solo los datos de entre las 8:00 y las 17:00 usando la función \texttt{subset} de R, debido a que la gran mayoría de surfistas practican el surf en esta franja horaria. Asimismo, hemos usado la función \texttt{sample\_n} de la librería \texttt{dplyr} de R para reducir aleatoriamente aún más los datos de cada \textit{surf spot} hasta los 500 \textit{data points}. Usando estos criterios, hemos reducido la cantidad de datos para Nazaré y Jaws desde los 35016 y 26704 periodos respectivamente hasta los 500 para cada uno.

\section{Análisis Descriptivo}%
\label{sec:metodos}

\begin{table}[htbp]
    \centering
    \begin{tabular}{lS[table-format=1.3]S[table-format=2.3]}
    \toprule
    & {Altura Ola (\si{\meter})} & {Velocidad Viento (\si{\knot})} \\
    \midrule
    Min.   & 0.0&  0   \\ 
    1er Qu.& 1.6   &  5   \\ 
    Mediana & 2.1   &  8   \\ 
    Media   & 2.414   &  8.666   \\ 
    3er Qu.& 3.0   & 11   \\ 
    Máx.   &11.4   & 38   \\ 
    \bottomrule
\end{tabular}

    \caption{summary Nazaré}
    \label{tab:summary_nazare}
\end{table}

El Cuadro \ref{tab:summary_nazare} muestra un resumen de los datos recogidos sobre Nazaré. Cómo se puede observar la altura máxima registrada es de 11.4 metros, a pesar de ello la media es de 2.4 metros debido a que durante la mayoría de días la altura está entre el mínimo (0 metros) y el primer cuartil (1.6 metros). Por este mismo motivo la mediana (de 2.1 metros) se encuentra por debajo de la media, y tercer cuartil es de 3 metros. Dicha distribución se puede ver con claridad en la Figura \ref{fig:hist_waves_nazare}.

\begin{figure}[H]
    \centering
    \includegraphics[height=5.6cm]{./figures/nazare_hist_wave.pdf}
    \caption{Histograma de olas en Nazaré}
    \label{fig:hist_waves_nazare}
\end{figure}

Con respecto a los datos sobre la velocidad del viento sucede algo parecido, pero en este caso los datos se concentran sobre todo entre la mediana (8 nudos) y el tercer cuartil (11 nudos) ya que la media es de 8.6 nudos. Sin embargo el valor máximo se aleja mucho de la media llegando a multiplicar su valor por más de cuatro siendo este de 38 nudos. También se puede observar que el mínimo, al igual que con la altura es de 0 nudos.


\begin{table}[htbp]
    \centering
    \begin{tabular}{lS[table-format=1.3]S[table-format=2.3]}
    \toprule
    & {Altura Ola (\si{\meter})} & {Velocidad Viento (\si{\knot})} \\
    \midrule
    Min.    &0.0      &  0    \\
    1er Cu.  &1.6    &  8    \\
    Mediana &1.9    & 12    \\
    Media   &2.047  & 11.46 \\
    3er Cu.  &2.4    & 15    \\
    Máx.    &5.9    & 32    \\
    \bottomrule
\end{tabular}

    \caption{summary Jaws}
    \label{tab:summary_jaws}
\end{table}


En el caso de Jaws, los datos se exponen en el Cuadro \ref{tab:summary_jaws}, y se observa que la distribución de la altura es bastante similar a la de Nazaré pero los datos están ligeramente más centrados (véase Figura \ref{fig:hist_waves_jaws}). En Jaws el rango de altura es menor ya que el valor mínimo sigue siendo de 0 metros pero el máximo en cambio es de 5.9, casi la mitad que en Nazaré. La media y la mediana prácticamente coinciden siendo 2 y 1.9 metros respectivamente. El primer cuartil se encuentra en 1.6 metros y el tercero en 2.4 metros.

\begin{figure}[H]
    \centering
    \includegraphics[height=5.6cm]{./figures/jaws_hist_wave.pdf}
    \caption{Histograma de olas en Jaws}
    \label{fig:hist_waves_jaws}
\end{figure}

La velocidad del viento en Jaws en comparación con la de Nazaré, a pesar de que el máximo (32 nudos) es inferior, es mucho superior con una media de 11.4 \si{\knot}. Tiene una distribución relativamente simétrica ya que la mediana se encuentra en 12 nudos y el primer y tercer cuartiles en 8 y 15 nudos respectivamente. El mínimo como es de esperar también es de 0 nudos.

\begin{figure}[H]
    \begin{subfigure}{.49\textwidth}
        \centering
        \includegraphics[width=0.9\textwidth]{./figures/boxplot_wind.pdf}
        \caption{Boxplot Viento}
        \label{fig:boxplot_wind}
    \end{subfigure}%
    \begin{subfigure}{.49\textwidth}
        \centering
        \includegraphics[width=0.9\textwidth]{./figures/boxplot_wave.pdf}
        \caption{Boxplot Olas}
        \label{fig:boxplot_wave}
    \end{subfigure}
    \caption{Boxplots}
    \label{fig:boxplots}
\end{figure}

En la Figura \ref{fig:boxplot_wind} podemos ver claramente como la media de la velocidad del viento en Jaws 
es superior a la de Nazaré, no obstante la variabilidad de Jaws es ligeramente superior como se puede apreciar 
en el primer cuartil. En cuanto a la altura de las olas (véase Figura \ref{fig:boxplot_wave}), la media 
es levemente mayor y a diferencia del viento, en este caso es Nazaré que tiene una variación ostensiblemente 
más amplia.

\section{Análisis Inferencial}%
\label{sec:resultados}
% Descriptiva más inferencia
Habiendo calculado la media de la altura de las olas tanto de Nazaré como de Jaws, hemos llegado a la
prueba de hipótesis \ref{eq:PH}.
\begin{align}
\label{eq:PH}
    H_0 : \mu_N &< \mu_J \nonumber \\
    H_1 : \mu_N &> \mu_J
\end{align}

\begin{align}
\label{eq:t}
    \hat{t} &= \frac{\bar{\mathcal{Y}_N} - \bar{\mathcal{Y}_J} }{
        S\sqrt{\nicefrac{1}{n_N} + \nicefrac{1}{n_J}}
    } \\
\label{eq:S2}
    S^2 &= \frac{\left(n_N - 1\right)S_N^2 + \left(n_J-1\right)S_J^2}{n_N+n_J-2}
\end{align}

Tras ejecutar la función \texttt{t.test} de R, pasándole como parámetros los datos de la altura de las olas 
de Nazaré y Jaws, hemos obtenido que el valor de \( \hat{t} \) es 1.6972 y el $p$-valor es 0.04499.
Paralelamente, hemos calculado  \( t_{n_N+n_J-2, 1-\nicefrac{\alpha}{2}} \), donde \(n_N = n_J = 500\)
y \(\alpha = 0.05\), obteniendo  que \( t_{998,\: 0.975} = 1.9623\). Por lo tanto, como \(\alpha > p-valor\) 
rechazamos nuestra hipótesis \(H_0\) y confirmamos que la media de la altura de las olas en Nazaré es
superior a Jaws.



%  	Welch Two Sample t-test
%  
%  data:  nazare_500$Wave and jaws_500$Wave
%  t = 1.6972, df = 984.19, p-value = 0.04499
%  alternative hypothesis: true difference in means is greater than 0
%  95 percent confidence interval:
%   0.003923912         Inf
%  sample estimates:
%  mean of x mean of y 
%     2.4562    2.3250 

%  
%  	Welch Two Sample t-test
%  
%  data:  nazare_500$Wave and jaws_500$Wave
%  t = 1.6972, df = 984.19, p-value = 0.955
%  alternative hypothesis: true difference in means is less than 0
%  95 percent confidence interval:
%        -Inf 0.2584761
%  sample estimates:
%  mean of x mean of y 
%     2.4562    2.3250 

%   	Welch Two Sample t-test
%   
%   data:  nazare_500$Wave and jaws_500$Wave
%   t = 1.6972, df = 984.19, p-value =  0.08998
%   alternative hypothesis: true difference in means is not equal to 0
%   95 percent confidence interval:
%    -0.02050262  0.28290262
%   sample estimates:
%   mean of x mean of y 
%      2.4562    2.3250 

\begin{equation}
\label{eq:IntConfi}
    IC\left( \mu_N - \mu_J, 1-\alpha\right) =
    \left[ \bar{x_N} - \bar{x_J} \pm t_{n_J+n_S-2, 0.975} \cdot S \sqrt{
        \frac{1}{n_N}  + \frac{1}{n_J}
    }
    \right]
\end{equation}

Usando la función \texttt{t.test} de R hemos obtenido:

\[IC\left( \mu_N - \mu_J, 0.95\right) = \left[ 0.003923912,  \infty \right]\]

\section{Análisis de Regresión Lineal Simple}
\label{sec:rls}

Tal como se explica en la sección \ref{sec:recogida_de_datos}, tomamos una muestra sobre nuestros datos
iniciales con el fin de poder realizar los cálculos de regresión con mayor facilidad. % IDK si aixo cal

\begin{align}
\label{eq:RLS}
    Y &= \text{Altura Ola}     \nonumber \\
    X &= \text{Velocidad Viento}     \nonumber \\
    Y &= \beta_1 + \beta_2 X 
\end{align}

\begin{equation}
\label{eq:correlation}
    \rho_{X,Y}=\mathrm{corr}(X,Y) = \frac{\mathrm{cov}(X,Y)}{\sigma_X \sigma_Y}
\end{equation}

Definimos pues nuestro modelo lineal como se muestra en \ref{eq:RLS}. I calculamos la regresión en R con
\texttt{lm}. Los resultados obtenidos se muestran en la tabla \ref{tab:results_RLS}.
En las Figuras \ref{fig:wind_waves_jaws_fit} y \ref{fig:wind_waves_nazare_fit} podemos observar la
regresión para Jaws y Nazaré respectivamente.

En el caso de Nazaré, obtenemos una correlación de 0.382916739203572 y en Jaws de 0.423654153616663. Existe
pues una correlación positiva entre el viento y la altura de las olas, ya que los valores de
correlación son mayores a 0. A pesar de esto, los valores de \(r^2\) son bastante bajos, 0.1795 y 0.1643 respectivamente, por lo que podemos concluir que pese a que su correlación es positiva, no es muy fuerte.

\begin{table}[htbp]
    \centering
    \begin{tabular}{lcc|cc}
        \toprule
        & \multicolumn{2}{c}{Nazaré} & \multicolumn{2}{c}{Jaws} \\
        \midrule
        & Estimado & Std.Error & Estimado & Std.Error \\
        \midrule
        \(\beta_1\)     & 1.5453684 & 0.11204219  &  1.3609970 & 0.10342377  \\
        \(\beta_2\)     & 0.1014741 & 0.01096999  &  0.1103989 & 0.01057748 \\
        \bottomrule
    \end{tabular}
    \caption{Resultados RLS}
    \label{tab:results_RLS}
\end{table}

%     Call:
%     lm(formula = Wave ~ Wind, data = jaws_500)
%     
%     Residuals:
%         Min      1Q  Median      3Q     Max 
%     -1.8962 -0.7442 -0.2130  0.4886  4.2582 
%     
%     Coefficients:
%                 Estimate Std. Error t value Pr(>|t|)    
%     (Intercept)  1.36100    0.10342   13.16   <2e-16 ***
%     Wind         0.11040    0.01058   10.44   <2e-16 ***
%     ---
%     Signif. codes:  0 ‘***’ 0.001 ‘**’ 0.01 ‘*’ 0.05 ‘.’ 0.1 ‘ ’ 1
%     
%     Residual standard error: 1.041 on 498 degrees of freedom
%     Multiple R-squared:  0.1795,	Adjusted R-squared:  0.1778 
%     F-statistic: 108.9 on 1 and 498 DF,  p-value: < 2.2e-16


\begin{figure}[H]
    \centering
    \includegraphics{./figures/jaws_500_lmFit.pdf}
    \caption{Regresión Velocidad de viento vs. Altura de olas en Jaws en Peahi, Hawái}
    \label{fig:wind_waves_jaws_fit}
\end{figure}

\begin{figure}[H]
    \centering
    \includegraphics{./figures/nazare_500_lmFit.pdf}
    \caption{Regresión Velocidad de viento vs. Altura de olas en Nazaré, Portugal}
    \label{fig:wind_waves_nazare_fit}
\end{figure}


\begin{figure}[H]
    \centering
    \includegraphics{./figures/jaws_all.pdf}
    \caption{Velocidad de viento vs. Altura de olas en Jaws en Peahi, Hawái}
    \label{fig:wind_waves_jaws_all}
\end{figure}

\begin{figure}[H]
    \centering
    \includegraphics{./figures/nazare_all.pdf}
    \caption{Velocidad de viento vs. Altura de olas en Nazaré, Portugal}
    \label{fig:wind_waves_nazare_all}
\end{figure}

\section{Discusión}%
\label{sec:discusión}

A modo de conclusión, hemos visto que la media de la altura de las olas en Nazaré es superior a la de Jaws,
2.414 y 2.047 metros respectivamente. De hecho, podemos afirmar que con una confianza del 95\% las olas en
Nazaré serán como mínimo 0.003924 metros más altas que en Jaws y por esta razón podemos confirmar
nuestra hipótesis inicial.

Además, como hemos apreciado en las Figuras \ref{fig:wind_waves_jaws_fit} y \ref{fig:wind_waves_nazare_fit}, la velocidad del viento sí que afecta a la altura de las olas, pues hemos obtenido una correlación para Nazaré y Jaws de 0.382916739203572 y 0.423654153616663 respectivamente. No obstante, el efecto no es tan fuerte y evidente como habíamos imaginado puesto que las correlaciones son bajas.

En la Figura \ref{fig:wind_waves_jaws_all} se puede observar nuestra regresión junto a la
curva \emph{smooth} generada por el R. Se aprecia una diferencia clara entre estas.
No hay un cambio apreciable en la altura de las olas, hasta que el viento supera los 10 nudos, por lo que
seria interesante realizar un análisis más en profundidad tomando solo los datos con viento entre 10 y 20 nudos.
Esta diferencia no se aprecia tanto en el caso de Nazaré (véase Figura \ref{fig:wind_waves_nazare_all}) por lo que 
podemos concluir que el efecto del viento es distinto en las dos localizaciones. Adicionalmente, 
los valores que hemos obtenido en la regresión lineal para \(\beta_2\) indican que el efecto del viento es
superior en Jaws pues la \(\beta_2\) de Nazaré es ligeramente inferior a la de Jaws.

\pagebreak
\appendix

\section{Código extracción de datos}%
\label{sec:codigo_extraccion_de_datos}

\lstinputlisting[language=Python]{./process.py}

\pagebreak

\section{Código R}%
\label{sec:codigo_r}

\lstinputlisting[language=R]{./notebookR.r}

\section{Gráficos Adicionales}
\label{sec:extra_plots}

% mn_diff <- mean(nazare_500$Wave) - mean(jaws_500$Wave)
% t.test(nazare_500$Wave, jaws_500$Wave, vas.equal=T, mu=mn_diff)

\begin{figure}[H]
    \centering
    \includegraphics{./figures/jaws_500_lm.pdf}
    \caption{Regresión Velocidad de viento vs. Altura de olas en Jaws}
    \label{fig:wind_waves_jaws_fit_anal}
\end{figure}

\begin{figure}[H]
    \centering
    \includegraphics{./figures/nazare_500_lm.pdf}
    \caption{Regresión Velocidad de viento vs. Altura de olas en Nazaré, Portugal}
    \label{fig:wind_waves_nazare_fit_anal}
\end{figure}

\begin{figure}[H]
    \centering
    \includegraphics[width=0.6\textwidth]{./figures/jaws_bin2d.pdf}
    \caption{Velocidad de viento vs. Altura de olas en Jaws en Peahi, Hawái}
    \label{fig:wind_waves_jaws}
\end{figure}

\begin{figure}[H]
    \centering
    \includegraphics[width=0.6\textwidth]{./figures/jaws_smooth.pdf}
    \caption{Velocidad de viento vs. Altura de olas en Jaws en Peahi, Hawái}
    \label{fig:jaws_smooth}
\end{figure}

\begin{figure}[H]
    \centering
    \includegraphics[width=0.6\textwidth]{./figures/nazare_bin2d.pdf}
    \caption{Velocidad de viento vs. Altura de olas en Nazaré, Portugal}
    \label{fig:wind_waves_nazare}
\end{figure}

\begin{figure}[H]
    \centering
    \includegraphics[width=0.6\textwidth]{./figures/nazare_smooth.pdf}
    \caption{Velocidad de viento vs. Altura de olas en Nazaré, Portugal}
    \label{fig:nazare_smooth}
\end{figure}

\end{document}

% vim:sw=2:ts=2:et:spell:spelllang=es: