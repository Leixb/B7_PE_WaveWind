\input{preamble}

%\geometry{margin=1in}

\title{
   Probabilidad y estadística - B7 \\
   \large 
   - Estudio del oleaje y el viento en Nazaré y Jaws -
}
\author{
  Aleix Boné \and
  Alex Herrero \and
  Albert Mercadé
}
\date{
  \today
}

\begin{document}
\maketitle
%\tableofcontents

\begin{abstract}
% resumen
% <= 250 palabras
\end{abstract}

\section{Introducción}%
\label{sec:introduccion}
% Justificación + objetivos
Dentro del ámbito del surf hay muchos factores a tener en cuenta en cuanto  a lo que se refiere a la ``ola perfecta''. Entre ellos, dos de los más importantes son el tamaño de la ola y la velocidad del viento.

En este caso queremos comparar dos de los surf spots\footnote{Lugar con olas surfeables} con las olas más grandes del mundo a lo largo de la historia; Nazaré en Portugal y Jaws en Peahi, Hawaii. Paralelamente observaremos qué relación tiene la velocidad de viento con el tamaño de las olas en ambos sitios.

\section{Recogida de datos}%
\label{sec:recogida_de_datos}

Para realizar nuestro análisis recogimos datos de viento y altura de olas de
dos localizaciones distintas. Escogimos dos playas conocidas por sus buenas
olas para hacer surf: Nazaré, Portugal y Jaws, Hawaii.

Pudimos obtener datos de oleaje y viento del archivo de \emph{WindGuru}
\footnote{\url{https://www.windguru.cz/archive.php}}. Con un script de
Python\footnote{El código se puede ver en el anexo
  \ref{sec:codigo_extraccion_de_datos}} extrajimos datos de la altura de las
olas y viento en periodos de 3 horas de 2006 hasta hoy. En el caso de Nazaré
obtuvimos datos de 4377 días (35016 periodos de 3 horas) y en Jaws 3338 días
(26704 periodos de 3 horas).

\section{Métodos}%
\label{sec:metodos}
% Métodos (unidades, variables, análisis)

\section{Resultados}%
\label{sec:resultados}
% Descriptiva más inferencia



\begin{figure}[!ht]
\label{fig:wind_waves_jaws}
  \caption{Velocidad de viento vs. Altura de olas en Jaws, Hawaii}
\centering
\includegraphics[width=0.6\textwidth]{./figures/jaws.pdf}
\end{figure}

\begin{figure}[!ht]
\label{fig:wind_waves_nazare}
  \caption{Velocidad de viento vs. Altura de olas en Nazaré, Portugal}
\centering
\includegraphics[width=0.6\textwidth]{./figures/nazare.pdf}
\end{figure}



\section{Discusión}%
\label{sec:discusión}


\pagebreak
\appendix

\section{Código extracción de datos}%
\label{sec:codigo_extraccion_de_datos}

\lstinputlisting[language=Python]{./process.py}

\end{document}

% vim:sw=2:ts=2:et:spell:spelllang=es:
