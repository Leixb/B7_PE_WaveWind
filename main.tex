\input{preamble}

\title{
   Probabilidad y estadística - B7 \\
   \large 
   - Estudio del oleaje y el viento en Nazaré y Jaws -
}
\author{
  Aleix Boné \and
  Alex Herrero \and
  Albert Mercadé
}
\date{
  \today
}

\begin{document}
\maketitle
%\tableofcontents

\begin{abstract}
% resumen
% <= 250 palabras
\end{abstract}

\section{Introducción}%
\label{sec:introduccion}
% Justificación + objetivos

\section{Recogida de datos}%
\label{sec:recogida_de_datos}

Para realizar nuestro análisis recogimos datos de viento y altura de olas de
dos localizaciones distintas. Escogimos dos playas conocidas por sus buenas
olas para hacer surf: Nazaré, Portugal y Jaws, Hawaii.

Pudimos obtener datos de oleaje y viento del archivo de \emph{WindGuru}
\footnote{\url{https://www.windguru.cz/archive.php}}. Con un script de
Python\footnote{El código se puede ver en el anexo
  \ref{sec:codigo_extraccion_de_datos}} extrajimos datos de la altura de las
olas y viento en periodos de 3 horas de 2006 hasta hoy. En el caso de Nazaré
obtuvimos datos de 4377 días (35016 periodos de 3 horas) y en Jaws 3338 días
(26704 periodos de 3 horas).

\section{Métodos}%
\label{sec:metodos}
% Métodos (unidades, variables, análisis)

\section{Resultados}%
\label{sec:resultados}
% Descriptiva más inferencia

\section{Discusión}%
\label{sec:discusión}


\pagebreak
\appendix

\section{Código extracción de datos}%
\label{sec:codigo_extraccion_de_datos}

\lstinputlisting[language=Python]{./process.py}



\end{document}

% vim:sw=2:ts=2:et:spell:spelllang=es:
