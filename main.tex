\input{preamble}

\definecolor{codegreen}{rgb}{0,0.6,0} \definecolor{codegray}{rgb}{0.5,0.5,0.5}
\definecolor{codepurple}{rgb}{0.58,0,0.82}
\definecolor{backcolour}{rgb}{0.95,0.95,0.92}

\lstdefinestyle{mystyle}{ backgroundcolor=\color{backcolour},
    commentstyle=\color{codegreen}, keywordstyle=\color{blue},
    numberstyle=\tiny\color{codegray}, stringstyle=\color{red},
    identifierstyle=\color{black}, basicstyle=\footnotesize,
    %breakatwhitespace=false,
    breaklines=true,
    %captionpos=b,                    keepspaces=true,
    numbers=left,                    numbersep=5pt,
    showspaces=false,
    %showstringspaces=false, showtabs=false,
    tabsize=4 }

\lstset{style=mystyle}


\graphicspath{ ./figures/ }


\title{
   Manolo
}
\author{
  Aleix Boné\\
  Alex Herrero\\
  Albert Mercadé
}
\date{
  \today
}

\begin{document}
\maketitle

\begin{abstract}
% resumen
% <= 250 palabras
\end{abstract}

\section{Introducción}%
\label{sec:introduccion}
% Justificación + objetivos

\section{Recogida de datos}%
\label{sec:recogida_de_datos}

Para realizar nuestro análisis recogimos datos de viento y altura de olas de
dos localizaciones distintas. Escogimos dos playas conocidas por sus buenas
olas para hacer surf: Nazaré, Portugal y Jaws, Hawaii.

Pudimos obtener datos de oleaje y viento del archivo de \emph{WindGuru}
\footnote{\url{https://www.windguru.cz/archive.php}}. Con un script de
python\footnote{El código se puede ver en el anexo
  \ref{sec:codigo_extraccion_de_datos}} extrajimos datos de la altura de las
olas y viento en periodos de 3 horas de 2006 hasta hoy. En el caso de Nazaré
obtuvimos datos de 4377 días (35016 periodos de 3 horas) y en Jaws 3338 días
(26704 periodos de 3 horas).

\section{Métodos}%
\label{sec:metodos}
% Métodos (unidades, variables, análisis)

\section{Resultados}%
\label{sec:resultados}
% Descriptiva más inferencia



\begin{figure}[h]
  \caption{Velocidad de viento vs. Altura de olas en Jaws, Hawaii}
\centering
\includegraphics[width=0.8\textwidth]{./figures/jaws.pdf}
\end{figure}

\begin{figure}[h]
  \caption{Velocidad de viento vs. Altura de olas en Nazaré, Protugal}
\centering
\includegraphics[width=0.8\textwidth]{./figures/nazare.pdf}
\end{figure}



\section{Discusión}%
\label{sec:discusión}

\pagebreak
\appendix

\section{Código extracción de datos}%
\label{sec:codigo_extraccion_de_datos}

\lstinputlisting[language=Python]{./process.py}

\end{document}

% vim:sw=2:ts=2:et:spell:spelllang=es:
